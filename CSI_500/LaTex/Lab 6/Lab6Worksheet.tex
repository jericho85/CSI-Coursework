% Assignment 2
\documentclass[11pt]{article}
\usepackage{fullpage}

%opening
\title{LaTex Worksheet 2}
\author{Jericho McLeod}

\begin{document}

\maketitle

\begin{abstract}
This is the assignment Worksheet 2 for the Spring-2019 semester of CSI-500.
\end{abstract}

\section{Literature Review}
The diffusion of innovation is the study of how, why, and the velocity of the spread of new ideas, products, and innovations a socially dessiminated. 


Innovation takes place via a process whereby a new 'thought, behavior, or thing' is conceived of and brought into reality. No innovation springs full-blown out of nothing; it must have antecedents. Diffusion of innovation occurs along a curve or statistical distribution where the initial participants are the most socially influential in the diffusion of innovation. This can be more valuable than traditional advertising, a fact that the marketing industry has noted and now is academically included in relevant curriculum. This is evident in the current market selection of social influencers, and can be tracked through new product releases through such individuals in social media: this agrees with the final point of the article, in that it shows the history of concerns expressed being mitigated by marketing innovators through specific action  \cite{Robertson67}.


Diffusion studies rely on the study of change in behavior rather than an observation at a fixed time, offering advantages over other types of research. However, focus had been lost on the value of innovations during the 1970s, as well as focus on well-conducted scientific study, seeing the rise of the misuse or ignoring of such factors as causality, unitary measurement, and other cognitive biases. The author noted that this was being overcome at the time by reapplying social structure in diffusion research via such methods as network analysis \cite{Rogers76}.


\section{Data}
As an example of the Diffusion of Innovation, one can consider the sales of consumer electronics. For this study, the data to be examined will be that of sales data for the iPad. 

\begin{tabular}{|r|c|}
	\hline 
	Fiscal Quarter & iPad Sales (M Units) \\
	\hline
	Jan-10 & 0.00\\
	\hline
	Mar-10 & 0.00\\
	\hline
	Jul-10 & 3.27\\
	\hline
	Oct-10 & 4.19\\
	\hline
	Jan-11 & 7.33\\
	\hline
	Apr-11 & 4.69\\
	\hline
	Jul-11 & 9.25\\
	\hline
	Oct-11 & 11.12\\
	\hline
	Jan-12 & 15.30\\
	\hline
	Apr-12 & 11.80\\
	\hline
	Jul-12 & 17.00 \\
	\hline
	Oct-12 & 14.04\\
	\hline
	Jan-13 & 22.86\\
	\hline
	Apr-13 & 19.48\\
	\hline
	Jul-13 & 14.62\\
	\hline
	Oct-13 & 14.08\\
	\hline
	Jan-14 & 26.04\\
	\hline
	Apr-14 & 16.35\\
	\hline
	Jul-14 & 13.28\\
	\hline
	Oct-14 & 12.32\\
	\hline
	Jan-15 & 21.42\\
	\hline
	Apr-15 & 12.62\\
	\hline
	Jul-15 & 10.93\\
	\hline
	Oct-15 & 8.88\\
	\hline
	Jan-16 & 16.12\\
	\hline
	Apr-16 & 10.25\\
	\hline
	Jul-16 & 9.95\\
	\hline
	Oct-16 & 9.27\\
	\hline
	Jan-17 & 13.08\\
	\hline
	Apr-17 & 8.90\\
	\hline
\end{tabular} 

\section{Methods}
Multiple mathematical models will be employed in the study of the diffusion of innovation. As an example of the types to be used, consider Equation 1 below.

\begin{eqnarray}
dx_{t} = x_{t} r \Big[
	1-\Big(
		\[\frac{x_{t}}{K}\]
	\Big)
\Big]
\end{eqnarray}
\\
\\

\bibliographystyle{plain}
\bibliography{Lab6WS}

\end{document}


% 