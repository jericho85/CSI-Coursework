%  Equation Example
\documentclass[11pt]{article}
\usepackage{fullpage}

%opening
\title{Lab 5 - LaTex Equation}
\author{Jericho McLeod}

\begin{document}

\maketitle

\begin{abstract}
In this lab, we use LaTex to do mathematical typesetting.
\end{abstract}

\section{Basics}
In this section we demonstrate some mathematical typsetting features of LaTex.

\subsection{In-line Expressions}
Algebra relies on variables and expressions, such as $z = x +y$.

\subsection{Super and Subscripts}
Here are some superscript and subscript examples: \\
$X^{3}, X^{Y^{2}} \\
Y_{1}, Y_{Z_{i}} \\
X^{i}_{j}, X_{j}^{i}$

\subsection{Fraction Examples}
Half of $N$ is $N/2$, but $N$ over $N+1$ is: \[\frac{n} {N+1}\]

\subsection{Root Examples}
The square root of 2 is $\sqrt{2}$, but the cube root of 2 is $\sqrt[3]{2}$.

\subsection{Ellipsis Examples}
Gauss's famous equation is" \[ \frac{N(N+1)}{2} = 1 + 2 + 3 + \cdots + N \] 
A sum of terms is $x_{1}, x_{2}, ~\ldots~, x_{N}$.

\subsection{Greek Letter Examples}
Here are some upper case Greek letters: $A, B, \Gamma, \Delta, \Upsilon$.\\
Here are some lower case Greek letters: $\alpha, \beta, \gamma, \delta, \upsilon$.\\

\subsection{Operator Examples}
Here are some mathematical operators: $5 \leq 8; 3 \geq 2; 6 \neq 8; 10 \ll 1000$.\\

\subsection{Summation Example}
Here is a summation operator:
\[ \sum_{i=0}^{i=10} x_{i}^{2} = 385\]\\

\subsection{Integral Example}
Here is an integration operator:
\[ \int_{-\infty}^{+\infty}\cos(x)dx=\sin(x)+C\]\\

\subsection{Trig Function Examples}
Here is a log operator:
\[f(x) = k\log(x) \]\\
Here are some trig functions:\\
\hspace{10mm}
$\sin \hspace{10mm}  \sec \hspace{10mm} \exp \hspace{10mm} \min \\
\cos \hspace{10mm} \cot \hspace{10mm} \inf \hspace{10mm} \max \\
\tan \hspace{10mm} \csc \hspace{10mm} \gcd \hspace{10mm} \lim $

\subsection{Array Examples}
Here is an array environment showing a 4x4 identity matrix $I$:
\[
	\begin{array}{cccc}
		1 & 0 & 0 & 0 \\
		0 & 1 & 0 & 0 \\
		0 & 0 & 1 & 0 \\
		0 & 0 & 0 & 1 \\
		\end{array}
\]
\\
\\
Here is an example of an equation array, called an eqnarray environment:

\begin{eqnarray}
x & = & (y + 2) (y-3) \\
x & = & y^{2} + (2y-3y)-6\\
x & = & y^{2}-y-6
\end{eqnarray}



\end{document}
